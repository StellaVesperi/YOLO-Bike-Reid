\documentclass{article}

% Language setting
% Replace `english' with e.g. `spanish' to change the document language
\usepackage[english]{babel}

% Set page size and margins
% Replace `letterpaper' with `a4paper' for UK/EU standard size
\usepackage[letterpaper,top=2cm,bottom=2cm,left=3cm,right=3cm,marginparwidth=1.75cm]{geometry}

% Useful packages
\usepackage{amsmath}
\usepackage{graphicx}
\usepackage[colorlinks=true, allcolors=blue]{hyperref}
\usepackage{xeCJK}
\usepackage{indentfirst}
\usepackage{hyperref}
\usepackage{float} 
\usepackage{tabularx}
\usepackage{booktabs}

\title{Bike Reid}
\author{罗浩铭~谢天~张芷苒}

\begin{document}
\maketitle

\begin{abstract}
本项目依托于YOLO目标检测算法,并结合监控数据,成功实现了校园内自行车与车主的实时监测和匹配。若系统检测到不匹配的人车组合,会立即发出警告并进行记录。这样,当车主的自行车遭遇失窃时,他们能迅速查询数据,从而更准确地定位到自行车的最后出现位置以及可能的嫌疑人。点击 \href{https://4332001876.github.io/YOLO_Bicycle_Theft_Detection/}{此处} 可以访问介绍我们项目实现过程的主页。

\end{abstract}

\section{需求分析}
\subsection{开发背景}

随着大学校园中自行车的日益普及,自行车盗窃事件也成为了一个日益严重的问题。对于忙于学业的大学生而言,一旦自行车遗失,寻找自行车或报备给保安部门查询监控成为了一项既耗时又无法保证效果的任务。为了更高效地帮助学生及时得知自行车的异常移动,并快速定位到疑似盗窃者和自行车的最后出现位置,我们采用了YOLO (You Only Look Once) 算法,创建了一个实时监测自行车是否遭到非法使用的系统。

在系统的初始学习阶段,我们利用校园内的监控资料,提取了每辆自行车与其骑手的图像,对其进行分类标记,确保了人车匹配的准确性。在后续的实时检测过程中,一旦系统侦测到之前未标记过的人车组合,它将自动标记并记录此次异常,并发出警报。若学生发现自己的自行车不翼而飞,他们可以直接在系统上上传自己自行车的照片。系统会立即返回历史捕捉到该自行车与不匹配骑者的监控记录及对应的地点,这不仅帮助车主迅速了解自行车的最后出现位置,还可以获取疑似盗窃者的图像,大大加速了寻找自行车的过程。

\subsection{市场分析}
\subsubsection{目标用户}

我们的项目针对的目标用户主要是大学校园内的学生和教职员工,他们每天都依赖自行车作为主要的出行工具。由于繁忙的学业和工作,这些用户需要一个简便且高效的方式来保障他们的自行车安全,特别是在面对日益增长的自行车盗窃事件时。他们需要的不仅是一个能够迅速通知自行车异常移动的系统,而且希望能在车辆失窃后,迅速定位到疑似盗窃者和自行车的最后出现位置,从而增加寻回失窃自行车的可能性。

\subsubsection{竞品分析}

在当前市场上,针对自行车盗窃问题,已有一些基于物理锁和GPS跟踪器组合的解决方案,旨在提供自行车的实时位置跟踪。但这些方案在校园内普及和推广上面临一定挑战。首先,它们往往需要用户购买和安装额外的硬件设备,这无疑增加了学生的经济负担。其次,由于校园内环境相对封闭、频繁的自行车停放变动,使得这些方案的效果并不稳定。更重要的是,这些竞品对于骑手的身份验证和识别功能几乎处于空白,这大大限制了其在防盗领域的实用性。相比之下,我们的系统利用了校园现有的监控设备,结合先进的YOLO算法,不仅能够实时跟踪自行车的位置,还具备对骑手身份的匹配和识别功能,无需用户额外投资或安装,为校园内的自行车安全提供了一个更为全面和便捷的解决方案,具有显著的竞争优势。

\subsection{主要功能}

\begin{enumerate}
   \item 提取频繁出现的人车图像进行匹配,将其认定为车与车主
   \item 实时监测当下的人车匹配情况
   \item 对与数据库中车-车主不符的匹配情况进行记录,并发出警报
\end{enumerate}

\section{概要设计}

天哥的部分。

\section{详细设计}

\subsection{总体设计}
本项目致力于校园安防,重点解决校园自行车的防盗追踪和自动预警问题,并提供了一种既保障隐私又实时高效的"BikeReid"方案。我们采用YOLOX目标检测算法,并以监控摄像头捕获的帧为输入,实时地识别、分割和追踪校园内正在骑行的自行车及其骑手。\\

考虑到监控画面的分辨率低、人像模糊以及自行车特征难以提取的问题,再加上多个监控位置拍摄带来的追踪难题,我们训练了专属于本项目的BikeReid工业级AI模型。此模型能有效地分离人与车的特征向量,并进行表征和度量学习,根据人的服饰、姿态及自行车的纹理等特征完成识别任务,这样避免了传统人脸识别可能导致的隐私问题。BikeReid在每帧的处理上,平均仅需0.024秒,确保了实时性。其在rank10的准确率高达79.53\%,满足工业部署的要求。我们还构建了一个MYSQL数据库,用于存储各自行车与骑手的配对、监控位置和时间戳等关键信息。核心的查询预警功能则利用Milvus引擎,能够对大量的特征向量进行高速相似度查询。一旦检测到新的配对,系统会将该配对和相应的画面加入预警库,为自行车盗窃发出预警,并协助快速找回失窃的自行车。在实际应用场景中,当某学生的自行车被盗,数据库会自动发送预警信息。学校方面只需进行一次查询,即可查看所有疑似的不匹配人车画面帧。选中特定画面后,再次查询将展示该自行车在监控记录中的所有历史配对图片,从而大大简化了查找过程,提高了寻回失窃自行车的效率。\\

此外,本项目具有很强的扩展性,适用于电瓶车、摩托车乃至汽车等其他交通工具的追踪,只需对BikeReid模型进行重新训练即可实现同样功能。此处,我们以难度最高的自行车识别作为示例进行演示。

\subsection{项目框架}

项目主要分为人车跨镜头重识别管线与数据库交互管线两部分。

\subsubsection{识别管线}

\begin{itemize}
    \item 我们的输入数据源为学校各处的监控画面。
    \item 利用YOLOx目标检测模型,我们可以从每一帧中准确分割并识别自行车与人的位置,进而定位其在画面中的坐标,并提供相应的锚框置信度。
    \item 为了确保识别的稳定性并避免受到监控摄像头的抖动、遮挡或跨摄像头的目标丢失等干扰,我们采用了deepsort动态追踪算法。
    \item 我们进一步将捕捉到的人与车的图像送入表示学习模型中,从中提取出人脸和自行车的特征向量。
    \item 在确定自行车与骑手的配对时,我们不仅考虑图像中的空间关系,还将关注二者的速度关联,确保配对的准确性(具体匹配算法仍在研究中)。为了增强这一步骤的稳定性,我们将使用一种滤波算法。
    \item 经过上述步骤后,我们会获得人脸与自行车的配对数据。通过再次应用滤波算法,我们会选取结果较为稳定的配对数据并存入数据库中。
\end{itemize}

\subsubsection{数据库交互与管理}

我们设立了一个核心数据库,专门存储人与自行车的特征向量配对信息。此外,为了加强盗窃预警功能,我们又构建了一个预警库,用于存放潜在可疑的人车配对及其对应的监控截图。一旦系统检测到新的或不常见的配对,该配对及相关画面即被录入预警库。此数据不仅起到预警作用,还能助力车主迅速找回被盗自行车。

关于数据库的操作与管理,以下是一些具体的策略与细节:

\begin{itemize}
    \item 在数据库初次运行时,我们需花费一定时间来积累并构建初始数据。特别地,每年新生入学时,我们建议暂停预警功能以便整合新生的数据。
    \item 考虑到学生可能购置新自行车导致新的人车配对产生,那些在预警库中反复出现且跨越一定时间跨度的配对,可以被认定为正常并加入到核心数据库中。
    \item 对于长时间未被访问或调用的数据,可以认为它们是由错误的识别产生或是毕业生的遗留数据,可以利用数据老化策略进行清理。
    \item 共享单车数据由于其特殊性应当从数据库中过滤出去,避免产生不必要的预警。
    \item 由于自然环境和技术因素的影响,识别结果可能存在不稳定性。因此,为了增强结果的鲁棒性,我们计划引入基于时间和地点的滤波算法,确保在相近的时间和地点,识别结果具有连续性和一致性。
\end{itemize}

\subsection{系统模块设计}

本系统由多个协同工作的模块组成,旨在确保自行车的安全性和可追踪性。以下详细介绍每个模块的功能和特点。

\subsubsection{输入模块}
该模块主要负责接收并处理学校各个监控点的视频数据。
\newline
\newline
\textbf{主要功能}:
\begin{itemize}
    \item 接受各监控点的实时视频流。
    \item 对视频流进行预处理,如分辨率调整、裁剪等。
    \item 将处理后的视频数据传输至下一个模块。
\end{itemize}

\subsubsection{目标检测模块}

本模块利用YOLOX-m模型,基于YOLOv8目标检测算法,对输入视频进行分析,准确地识别并定位自行车和人。
\newline
\newline
\textbf{主要功能}:
\begin{itemize}
    \item 分析视频数据,识别其中的自行车和行人。
    \item 仅保留高置信度的目标以降低误识别率。
    \item 输出所识别目标的位置信息。
\end{itemize}

\subsubsection{表示学习模块}

该模块深度分析视频中的人和自行车图像,提取其关键特征向量。
\newline
\newline
\textbf{主要功能}:
\begin{itemize}
    \item 接收并处理目标检测模块传来的图像数据。
    \item 从图像中提取人脸和自行车的特征向量。
    \item 将特征向量传输至下一模块。
\end{itemize}

\subsubsection{匹配模块}
利用先进的算法,该模块将人与特定的自行车进行匹配。
\newline
\newline
\textbf{主要功能}:
\begin{itemize}
    \item 接收表示学习模块的特征向量。
    \item 根据空间和速度关系进行人与自行车的匹配。
    \item 输出匹配后的人车配对结果。
\end{itemize}

\subsubsection{数据库模块}
该模块使用mysql和milvus来管理和存储关系数据和特征向量。
\newline
\newline
\textbf{主要功能}:
\begin{itemize}
    \item 利用mysql存储和管理关系数据。存储字段包括:
    \begin{itemize}
        \item pair\_id (主键, 创建索引,与milvus中的pair\_id相同)
        \item bicycle\_id (外键, 关联至bicycle\_photos表)
        \item person\_id (外键, 关联至person\_photos表)
        \item create\_date (创建日期)
        \item camera\_id (摄像机ID)
        \item modify\_date (修改日期)
    \end{itemize}
    \item 使用milvus存储和查询特征向量。存储字段包括:
    \begin{itemize}
        \item pair\_id (主键, 创建索引)
        \item bicycle\_embedding (自行车特征向量)
        \item person\_embedding (人特征向量)
    \end{itemize}
    \item 提供接口供其他模块调用和查询。
\end{itemize}

\subsubsection{预警模块}
该模块监测新的人车配对并发出预警。
\newline
\newline
\textbf{主要功能}:
\begin{itemize}
    \item 实时检测并分析新的人车配对。
    \item 将可疑配对及其视频数据加入预警库。
    \item 发出潜在的自行车盗窃预警。
\end{itemize}

\subsubsection{接口模块}
为确保系统与外部其他系统的高效协同,该模块提供了一系列的接口功能。
\newline
\newline
\textbf{主要功能}:
\begin{itemize}
    \item 提供与监控系统的视频数据交互接口。
    \item 与校园安全系统实现联动。
    \item 提供API以与其他外部系统进行数据交换。
\end{itemize}

此模块设计确保了系统的高效、稳定和可扩展性,从而为校园自行车安全提供了全面的保障。

\section{测试与改进}
罗神来补充,图片发给我贴进去。

\subsection{Re-ID模型评估结果}

\subsubsection{预训练Resnet-18模型直接用于人车对}
\begin{table}[H]
\centering
\begin{tabular}{l|cccccc}
\toprule
Dataset & Rank-1 & Rank-5 & Rank-10 & mAP & mINP & metric \\
\midrule
BikePerson & 0.12 & 0.65 & 1.22 & 0.19 & 0.09 & 0.15 \\
\bottomrule
\end{tabular}
\end{table}

\subsubsection{VehicleID模型直接用于人车对}
\begin{table}[H]
\centering
\begin{tabular}{l|cccccc}
\toprule
Dataset & Rank-1 & Rank-5 & Rank-10 & mAP & mINP & metric \\
\midrule
BikePerson & 0.97 & 2.47 & 3.71 & 0.60 & 0.13 & 0.78 \\
\bottomrule
\end{tabular}
\end{table}

\subsubsection{Market1501 People Re-ID模型直接用于人车对}
\begin{table}[H]
\centering
\begin{tabular}{l|cccccc}
\toprule
Dataset & Rank-1 & Rank-5 & Rank-10 & mAP & mINP & metric \\
\midrule
BikePerson & 0.93 & 2.55 & 3.80 & 0.62 & 0.13 & 0.77 \\
\bottomrule
\end{tabular}
\end{table}

\subsubsection{Market1501 People Re-ID模型训练后用于人车对}
\begin{table}[H]
\centering
\begin{tabular}{l|cccccc}
\toprule
Dataset & Rank-1 & Rank-5 & Rank-10 & mAP & mINP & metric \\
\midrule
BikePerson & 67.95 & 81.73 & 88.68 & 66.71 & 51.75 & 67.33 \\
\bottomrule
\end{tabular}
\end{table}

\subsubsection{Market1501 People Re-ID模型训练后用于自行车}
\begin{table}[H]
\centering
\begin{tabular}{l|cccccc}
\toprule
Dataset & Rank-1 & Rank-5 & Rank-10 & mAP & mINP & metric \\
\midrule
BikePerson & 56.25 & 71.43 & 79.53 & 52.79 & 34.39 & 54.52 \\
\bottomrule
\end{tabular}
\end{table}

\subsection{}

\subsection{风险管理计划}
\subsubsection{关于识别精度的风险及其对策}

在自行车识别中,工作环境的严苛性要求在高噪音、低分辨率、少量特征点以及多样化的拍摄角度下实现准确的车辆识别。尽管车辆图像可以通过YOLO技术分割,从而有效降低背景的干扰,但为确保准确度,我们需要采纳高精度的表示学习方法以区分自行车的身份。而对于人脸的部分,可以直接使用现有的库进行识别。

需要注意的是,识别结果可能不稳定,同一辆车在相近的时间和地点下的识别结果可能会产生波动。为了增加稳定性和鲁棒性,我们可能需要引入基于时间和地点信息的滤波算法,确保在时间和地点的小范围变动内,识别结果保持连贯。\\

\textbf{提高识别精度的策略如下:}
\begin{itemize}
    \item \textbf{改进表示学习方法}:研究并采用更高精度的表示学习方法,例如采用更深、更复杂的神经网络结构。同时,考虑利用预训练模型和迁移学习,从大规模数据集中提取有价值的先验知识。
    
    \item \textbf{数据增强技术}:利用旋转、平移、缩放等数据增强方法,扩充和丰富自行车图像数据,以增强模型的泛化能力。
    
    \item \textbf{设置合理阈值}:在目标检测环节,适当设置置信度阈值和非极大值抑制(NMS)参数,确保高置信度目标的准确识别,从而降低误识别率。
    
    \item \textbf{滤波算法}:采用基于时间和地点的滤波技术,保持短时间内识别结果的稳定性,充分考虑车辆在连续帧间的动态关系。
    
    \item \textbf{定期校准}:定期进行系统校准,对自行车和人脸识别模型进行评估和微调,通过标注良好的样本进行测试和验证,确保系统的高准确度。
    
    \item \textbf{持续监控与优化}:实时监控系统表现,根据用户反馈和日志数据进行分析,修复潜在问题,并根据实际反馈持续优化算法和参数。
    
    \item \textbf{备用与应急措施}:准备应对识别精度下降或其他故障的备选方案,如预设人工审核环节或手动标注,确保及时纠正错误并恢复系统正常运行。
\end{itemize}

\section{用户指南}
12点ddl之前是否可以考虑省去这一部分?

\section{项目总结}
\subsection{任务分配}

\begin{table}[H]
\centering
\begin{tabular}{l|r}
姓名 & 贡献 \\\hline
罗浩铭 & 00 \\
谢天 & 00 \\
张芷苒 & 00
\end{tabular}
\caption{\label{tab:widgets}任务分配表}
\end{table}

\subsection{遇到的困难}

你们来写一点吧。

\subsection{开发感悟}

天哥来扯一点。可以根据给分表讲讲可移植性什么的...



\end{document}
